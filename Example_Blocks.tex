\begin{question}{Question 1:}
  For all $x$, determine the coefficients in the FIR filter that result in a filter that performs as you'd expect a FIR filter to work.
\end{question}

\begin{answer}{Answer 1:}
Yo Shawty, that jacket is tighhht son, yaaah meaaan. Right, left, Right, left Hughhhh sonn Hughhh. Right, left, right, left Hughh sonn Hughh.
\end{answer}

\begin{example}{Example 1:}
I am a world before I am a Man. I was a creature before i could stand. I will remember before i forget. BEFORE I FORGET;
\end{example}

\begin{codeblock}{Example Codeblock for MATLAB}
Here is some example code you might like to work with; this code does nothing important :) :) :) :) :) :) :)\\

Listing \ref{lst:Matlab} shows an interesting example of code within matlab.
\begin{lstlisting}[style=Matlab-editor,basicstyle=\mlttfamily,caption={MATLAB sample} \label{lst:Matlab}]
%% displaying some quantisation error
signedness=1;   % 1=signed, 0=unsigned;
intbits=5;      % 2^5   = 32
fractbits=1;   % 2^-10 =
wordLen=signedness+intbits+fractbits
q = fixed.Quantizer(signedness,wordLen,fractbits, ...
  'Nearest','Saturate')

for x=0:1:20
  sin(x)+1
end

t=0:.01:100;
x=(10*sin(t)).*(3*cos(2*t));
x_f = fi(x,signedness,wordLen,fractbits);


%{
  This is just a multiline
  comment to see this command works
%}
\end{lstlisting}
\end{codeblock}



\begin{codeblock}{Example Codeblock for Python}
In listing \ref{lst:Python} the python language pack for listing is explored.

\begin{lstlisting}[language=python,basicstyle=\mlttfamily,caption={Python Sample} \label{lst:Python}]
string1 = "Linux"
string2 = "Mint"
joined_string = string1 + string2
print(joined_string)

# Assign a numeric value
number = 70

# Check the is more than 70 or not
if (number >= 70):
    print("You have passed")
else:
    print("You have not passed")
\end{lstlisting}
\end{codeblock}
\clearpage
