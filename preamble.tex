%Define the document%
\documentclass[12pt]{article}
\usepackage[T1]{fontenc}
\usepackage[a4paper, margin=2cm]{geometry}
\usepackage{amsmath}
\usepackage{amsfonts}
\usepackage{graphicx}
\usepackage[none]{hyphenat}
\usepackage[table]{xcolor} %this might be redundant' as xcolor loaded later
\usepackage{enumitem}
\usepackage{float}
\usepackage{color, soul} %allows todo highlighting
\usepackage[ddmmyyyy]{datetime}
\usepackage{xcolor}
\usepackage[framemethod=tikz]{mdframed}
\usepackage{listings}
\usepackage{matlab-prettifier}

\renewcommand{\contentsname}{Table Of Contents}
%\renewcommand{\listfigurename}{Figures}
%\renewcommand{\listtablename}{Tables}

%Fontset Changing%
%\usepackage[defaultfam,extralight,tabular,lining]{montserrat} % Use Montserrat Font.
\renewcommand{\familydefault}{\sfdefault} % Sans-Serif Default family (similar to helvetica)

% Bibliography style; suitable for UniSA
\usepackage{natbib}
\bibpunct{(}{)}{;}{a}{}{,} %Sets the punctuation for UniSA Style
\bibliographystyle{/Users/kobrien/Documents/Study/unisa2020}

% Remove the noindent
\setlength{\parindent}{0em}

% Set some common math letters
\newcommand\R{$\mathbb{R}$}
\newcommand\I{$\mathbb{I}$}
\newcommand\C{$\mathbb{C}$}
\newcommand\Z{$\mathbb{Z}$}
% Define the colours for block styles.
\mdfdefinestyle{bluestyle}{%
linecolor=blue!100!,outerlinewidth=1pt,%
frametitlerule=true,frametitlefont=\sffamily\bfseries\color{white},%
frametitlerulewidth=1pt,frametitlerulecolor=blue!100,%
frametitlebackgroundcolor=blue!100!,
backgroundcolor=blue!5!,
innertopmargin=\topskip,
roundcorner=4pt
}

\mdfdefinestyle{redstyle}{%
linecolor=red!100,outerlinewidth=1pt,%
frametitlerule=true,frametitlefont=\sffamily\bfseries\color{white},%
frametitlerulewidth=1pt,frametitlerulecolor=red!100,%
frametitlebackgroundcolor=red!100,
backgroundcolor=red!10!,
innertopmargin=\topskip,
roundcorner=4pt
}

\mdfdefinestyle{greenstyle}{%
linecolor=green!60!black,outerlinewidth=1pt,%
frametitlerule=true,frametitlefont=\sffamily\bfseries\color{white},%
frametitlerulewidth=1pt,frametitlerulecolor=green!60!black,%
frametitlebackgroundcolor=green!60!black,
backgroundcolor=green!10!,
innertopmargin=\topskip,
roundcorner=4pt
}

\mdfdefinestyle{graystyle}{%
linecolor=gray,outerlinewidth=1pt,%
frametitlerule=true,frametitlefont=\sffamily\bfseries\color{white},%
frametitlerulewidth=1pt,frametitlerulecolor=gray,%
frametitlebackgroundcolor=gray,
backgroundcolor=lightgray!20,
innertopmargin=\topskip,
roundcorner=4pt
}

% Create new environments using colour blocks (Exmaple, Question, Answer etc.)

\newmdenv[style=bluestyle]{quest}
\newenvironment{question}[1]
  {\begin{quest}[frametitle=#1]}
  {\end{quest}}

\newmdenv[style=redstyle]{ans}
\newenvironment{answer}[1]
  {\begin{ans}[frametitle=#1]}
  {\end{ans}}

\newmdenv[style=greenstyle]{code}
\newenvironment{codeblock}[1]
  {\begin{code}[frametitle=#1]}
  {\end{code}}

\newmdenv[style=graystyle]{ex}
\newenvironment{example}[1]
  {\begin{ex}[frametitle=#1]}
  {\end{ex}}
