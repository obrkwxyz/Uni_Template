%!TEX root = ../Main.tex

\begin{codeblock}{rgb2ycb.m -- RGB => YCbCr Transform}
This function provides optimised conversion of an RGB to YCbCr conversion of a ($m,n,3$) RGB double matrix.
    
\begin{lstlisting}[style=Matlab-editor,basicstyle=\mlttfamily, caption={}\label{lst:rgb2ycb}]
   
function  [im_ycbcr] = rgb2ycb(improc)
%     custom RGB to YCbCr Image Transform
%     Usage:
%     rgb2ycb2(image); Image is [R;G;B] (m,n,3) double matrix
%     Returns 3x rgb image, (m,n,3) double matrix [Y;0,0], [128,Cb,0], 
%     [128,0,Cr]
%     The function conversion to YCbCr does not comply with ITU-R BT.601
%     however is sufficient, and requested in this form as per the 
%     assignment sheet
%     Nested for-loops are utilised to within this script, noting in 
%     previous matlab versions this was a performance penalty compared to 
%     elementwise multiplication; this is supposedly not the case in modern
%     version; however it may be worth implementing both methods and seeing
%     how memory moves occur / potential gains

% RGB2YCbCr Transform Matrix;
% ITU-R BT.601 : 2.51, 2.5.2 Normalised Transformation 
T = [0.299      0.587      0.114;    ...
    -0.1687359 -0.3312641  0.5;      ...
     0.5       -0.4186875 -0.0813124]; 

[x,y,~] = size(improc);
im_ycbcr = zeros(x,y,3);

for i =  1:x
    for j = 1:y
    im_ycbcr(i,j,:) = T * [improc(i,j,1);improc(i,j,2);improc(i,j,3)];
%   Add any offset required; 
    im_ycbcr(i,j,:) = im_ycbcr(i,j,:) + cat(3,[0],[0],[0]); 
    end
end
end

\end{lstlisting}
\end{codeblock}
\clearpage

\begin{codeblock}{rgb2ycb2.m -- RGB => YCbCr Colourised Transform}
    This function provides colourised conversion of an RGB to YCbCr conversion of a ($m,n,3$) RGB double matrix.
    Usual chrominance layers are one-dimensional and effectively show grayscale images; this script shows  false coloured chrominance to highlight the differences
        
    \begin{lstlisting}[style=Matlab-editor,basicstyle=\mlttfamily, caption={}\label{lst:rgb2ycb2}]
function  [y,cb,cr] = rgb2ycb2(improc)
%     Custom RGB to YCbCr Colourised Image Transform
%     Usage:
%     rgb2ycb2(image); Image is [R;G;B] (m,n,3) double matrix
%     Returns 3x rgb image, (m,n,3) double matrix [Y;0,0], [128,Cb,0], 
%     [128,0,Cr]
%     The function conversion to YCbCr does not comply with ITU-R BT.601
%     however is sufficient, and requested in this form as per the 
%     assignment sheet
%     Nested for-loops are utilised to within this script, noting in 
%     previous matlab versions this was a performance penalty compared to 
%     elementwise multiplication; this is supposedly not the case in modern
%     version; however it may be worth implementing both methods and seeing
%     how memory moves occur / potential gains

% RGB2YCbCr Transform Matrix;
% ITU-R BT.601 : 2.51, 2.5.2 Normalised Transformation 
T = [0.299      0.587      0.114;    ...
    -0.1687359 -0.3312641  0.5;      ...
     0.5       -0.4186875 -0.0813124]; 

[x,y,~]  = size(improc);
im_ycbcr = zeros(x,y,3);

% transform the image
for i =  1:x
    for j = 1:y
        im_ycbcr(i,j,:) = T * [improc(i,j,1);improc(i,j,2);improc(i,j,3)];
    end
end

zmat = zeros(x,y);
omat = 0.5*ones(x,y);
y_s  = cat(3, im_ycbcr(:,:,1), zmat,zmat);
cb_s = cat(3, omat, im_ycbcr(:,:,2), zmat);
cr_s = cat(3, omat, zmat, im_ycbcr(:,:,3));
y    = ycb2rgb(y_s);
cb   = ycb2rgb(cb_s);
cr   = ycb2rgb(cr_s);

end

\end{lstlisting}
\end{codeblock}


\begin{codeblock}{ycb2rgb.m -- YCbCr => RGB Transform}
    This function provides conversion of a YCbCr to RGB conversion of a ($m,n,3$) RGB double matrix.
        
    \begin{lstlisting}[style=Matlab-editor,basicstyle=\mlttfamily, caption={}\label{lst:ycb2rgb}]
function  [im_rgb] = ycb2rgb(improc)
%     custom YCbCr to RGB Image Transform
%     Usage:
%     rgb_YCbCr(image); Image is [R;G;B] (m,n,3) double matrix
%     Returns YCbCr image, (m,n,3) double matrix [Y;C;B]
%     The function conversion to YCbCr does not comply with ITU-R BT.601
%     however is sufficient, and requested in this form as per the 
%     assignment sheet.
%     Nested for-loops are utilised to within this script, noting in 
%     previous matlab versions this was a performance penalty compared to 
%     elementwise multiplication; this is supposedly not the case in modern
%     version; however it may be worth implementing both methods and seeing
%     how memory moves occur / potential gains

% RGB2YCbCr Transform Matrix;
% ITU-R BT.601 : 2.51, 2.5.2 Normalised Transformation 
T = [0.299      0.587      0.114;    ...
    -0.1687359 -0.3312641  0.5;      ...
    0.5        -0.4186875 -0.0813124]; 

% improc = (double(improc))/256;
[x,y,~] = size(improc);
im_rgb   = zeros(x,y,3);

for i =  1:x
    for j = 1:y
%     im_rgb(i,j,:) = T^-1 * [improc(i,j,1); improc(i,j,2); improc(i,j,3)]; 
%     Add any offset required; 
%     im_rgb(i,j,:)   = im_rgb(i,j,:)   + cat(3,[0],[0],[0]);    
%     im_rgb          = min(1,max(0,im_rgb));
%{        
    Use of min(max()) to limit range (0,1) results in a SERIOUSLY 
    inefficient
    script, using un-optimised parts of the CPU for comparison, opposed to
    logical operation (0<=ph<=1). This results in processing taking ~ 2
    minutes on a Ryzen9-5650x CPU, 100% Usage @ 4.4GHz. Further, its
    addressing  the memory twice - ie Two-pass-process. This is improved 
    and runs much faster by placeholding the value as below. Alternatively
    working in uint8/uint16 might allow access to different hardware ALU 
    in the CPU.
%}        
        % Optimised transform
        placehold =  T^-1 * [improc(i,j,1); improc(i,j,2); improc(i,j,3)]; 
        if placehold <= 0
            placehold = 0;
        elseif placehold >= 1
            placehold = 1;
        end
        im_rgb(i,j,:) = placehold;
    end
end

    \end{lstlisting}
\end{codeblock}




\begin{codeblock}{rgb2y.m -- RGB => Y Transform}
    This function provides colourised conversion of an RGB to Y conversion of a ($m,n,3$) RGB double matrix. In many cases we are not interested in the chrominance layers, and hence discard.

        
    \begin{lstlisting}[style=Matlab-editor,basicstyle=\mlttfamily, caption={}\label{lst:rgb2y}]
function  [im_y] = rgb2y(improc)
%     custom RGB to Y Image Transform
%     Usage:
%     rgb2y(image); Image is [R;G;B] (m,n,3) double matrix
%     Returns 1x Y image, (m,n,1) double matrix [Y]
%     The function conversion to YCbCr does not comply with ITU-R BT.601
%     however is sufficient, and requested in this form as per the 
%     assignment sheet

% RGB2YCbCr Transform Matrix;
% ITU-R BT.601 : 2.51, 2.5.2 Normalised Transformation 
T = [0.299      0.587      0.114;    ...
    -0.1687359 -0.3312641  0.5;      ...
     0.5       -0.4186875 -0.0813124]; 

[x,y,~] = size(improc);
im_ycbcr = zeros(x,y,3);

for i =  1:x
    for j = 1:y
    im_ycbcr(i,j,:) = T * [improc(i,j,1);improc(i,j,2);improc(i,j,3)];
%   Add any offset required; 
    im_ycbcr(i,j,:) = im_ycbcr(i,j,:) + cat(3,[0],[0],[0]); 
    end
end
im_y = im_ycbcr(:,:,1);
end
\end{lstlisting}
\end{codeblock}
\clearpage

\begin{codeblock}{ds\_2h1v.m -- Chrominance Downsample 2H1V}
    This function provides chrominance downsampling 2Horizontal, 1Vertical. It accepts, and returns an image of ($m,n,3$) however the luminance channel is un-modified.
        
    \begin{lstlisting}[style=Matlab-editor,basicstyle=\mlttfamily, caption={}\label{lst:ds2h1v}]
function [im_ds2h1v] = ds_2h1v(improc)
%     Custom Chrominance Downsampling (2h1v)
%     Usage:
%     ds_2h1v(image); Image is [Y;Cb;Cr] (m,n,3) double matrix
%     It would be preferred to pass in a RGB image and handle the YCbCr
%     translation within this script but eh.

%separate the layers
Y   = improc(:,:,1);Cb  = improc(:,:,2);Cr  = improc(:,:,3); 

xstep = 2; h = xstep-1;
ystep = 1; v = ystep-1;
[y,x] = size(Cb);

% Y = Y; -- Dont adjust the luminance layers
% Within the loop

for i = 1:ystep:y
    for j = 1:xstep:x
        Cb(i:i+v,j:j+h) = mean(mean(Cb(i:i+v,j:j+h)));
        Cr(i:i+v,j:j+h) = mean(mean(Cr(i:i+v,j:j+h)));

    end
end
im_ds2h1v = cat(3,Y,Cb,Cr);


    \end{lstlisting}
\end{codeblock}

\begin{codeblock}{ds\_2h2v.m -- Chrominance Downsample 2H2V}
    This function provides chrominance downsampling 2Horizontal, 2Vertical. It accepts, and returns an image of ($m,n,3$) however the luminance channel is un-modified.
        
    \begin{lstlisting}[style=Matlab-editor,basicstyle=\mlttfamily, caption={}\label{lst:ds2h2v}]
function [im_ds2h2v] = ds_2h2v(improc)
%     Custom Chrominance Downsampling (2h2v)
%     Usage:
%     ds_2h1v(image); Image is [Y;Cb;Cr] (m,n,3) double matrix
%     It would be preferred to pass in a RGB image and handle the YCbCr
%     translation within this script but eh.

Y   = improc(:,:,1);Cb  = improc(:,:,2);Cr  = improc(:,:,3); %separate the layers

xstep = 2; h = xstep-1;
ystep = 2; v = ystep-1;
[y,x] = size(Cb);

% Y = Y; -- Dont adjust the luminance layers
% Within the loop

for i = 1:ystep:y
    for j = 1:xstep:x
        Cb(i:i+v,j:j+h) = mean(mean(Cb(i:i+v,j:j+h)));
        Cr(i:i+v,j:j+h) = mean(mean(Cr(i:i+v,j:j+h)));
    end
end
im_ds2h2v = cat(3,Y,Cb,Cr);
    \end{lstlisting}
\end{codeblock}

\begin{codeblock}{ds\_4h4v.m -- Chrominance Downsample 4H4V}
    This function provides chrominance downsampling 4Horizontal, 4Vertical. It accepts, and returns an image of ($m,n,3$) however the luminance channel is un-modified.
        
    \begin{lstlisting}[style=Matlab-editor,basicstyle=\mlttfamily, caption={}\label{lst:ds4h4v}]
function [im_ds4h4v] = ds_4h4v(improc)
%     Custom Chrominance Downsampling (4h4v)
%     Usage:
%     ds_2h2v(image); Image is [Y;Cb;Cr] (m,n,3) double matrix
%     It would be preferred to pass in a RGB image and handle the YCbCr
%     translation within this script but eh.

Y   = improc(:,:,1);Cb  = improc(:,:,2);Cr  = improc(:,:,3); %separate the layers

xstep = 4; h = xstep-1;
ystep = 4; v = ystep-1;
[y,x] = size(Cb);

% Y = Y; -- Dont adjust the luminance layers
% Within the loop

for i = 1:ystep:y
    for j = 1:xstep:x
        Cb(i:i+v,j:j+h) = mean(mean(Cb(i:i+v,j:j+h)));
        Cr(i:i+v,j:j+h) = mean(mean(Cr(i:i+v,j:j+h)));
    end
end
im_ds4h4v = cat(3,Y,Cb,Cr);
    \end{lstlisting}
\end{codeblock}        
\clearpage

\begin{codeblock}{ds\_dct.m -- Discrete Cosin Transform Downsampling}
    This function provides the main frequency-domain functionality of the JPEG compression algorithm with included quantisation. The accepted input is ($m,n,3$) and expected the range is (0,1) double. This function re-scales the image, and DCT downsampling of the luminance layer, however returns the chrominance layers unmodified.
        
    \begin{lstlisting}[style=Matlab-editor,basicstyle=\mlttfamily, caption={}\label{lst:dsdct}]
function [im_dct] = ds_dct(improc)
%     Custom Discrete Cosine Transform Downsampling
%     Usage:
%     [im_dct] = ds_dct(image); 
%     Image is [Y;Cb;Cr] (m,n,3) double matrix
%     Returns:
%     im_dct (m,n,3) in YCbCr format with
%     Y layer quantised and compressed via Jpeg methods

%separate the layers
Y   = improc(:,:,1);Cb  = improc(:,:,2);Cr  = improc(:,:,3);
Y   = Y*256;
Q= [8  7  7  7  8  9  11 13; ...
    7  7  7  7  8  9  11 14; ...
    7  7  8  8  9  11 13 16; ...
    7  7  8  10 12 14 16 20; ...
    8  8  9  12 15 18 22 26; ...
    9  9  11 14 18 23 29 36; ...
    11 11 13 16 22 29 38 49; ...
    13 14 16 20 26 36 49 65];

[y,x] = size(Y);
X     = zeros(y,x);
X_    = zeros(y,x);
X_p   = zeros(y,x);
X_hat = zeros(y,x);

xstep = 8; h = xstep-1;
ystep = 8; v = ystep-1;

for i = 1:ystep:y
    for j = 1:xstep:x
        X(i:i+v,j:j+h)    = dct(dct(Y(i:i+v,j:j+h))')';
        X_(i:i+v,j:j+h)  = round(X(i:i+v,j:j+h)./Q);
        X_p(i:i+v,j:j+h) = X_(i:i+v,j:j+h).*Q;
        X_hat(i:i+v,j:j+h)   = idct(idct(X_p(i:i+v,j:j+h))')';
    end
end
X_hat  = X_hat/256;
im_dct = cat(3,X_hat,Cb,Cr);
end

    \end{lstlisting}
\end{codeblock}

\begin{codeblock}{imsad.m -- Sum of Absolute Difference}
    This function provides a quick reference call to calculate the sum of absolute differences between two matricies as it is called regularly in the code; it assumes both input matrix are the same size ($m,n,1$) and returns a scalar quantity.
        
    \begin{lstlisting}[style=Matlab-editor,basicstyle=\mlttfamily, caption={imsad(imref, impred)}\label{lst:imsad}]       
function [im_sad] = imsad(imref,impred)
    %     Custom Sum of Absolute Differences
    %     Usage:
    %     [im_sad] = imsad(image); 
    %     Image is [Y] (m,n,1) double matrix
    %     Returns:
    %     Sum of Absolute Difference
    %     \sum{\sum{\abs{a-b}}}

im_sad = sum(sum(abs(imref-impred))); 
\end{lstlisting}
\end{codeblock}

\begin{codeblock}{DS Sample Testing.m -- Down sample testbench}
This file produces image patterns that demonstrate effective downsampling.
        
\begin{lstlisting}[style=Matlab-editor,basicstyle=\mlttfamily, caption={}\label{lst:dstest}]
% DS sample testing
% Downsampling test bench
ay  = [1:1:32; 32:-1:1; 1:16, 1:16];
acb = [1:32;1:16,16:-1:1; 1:32];
acr = [1:16,1:16; 1:32; 32:-1:1];
a = repmat(cat(3,ay,acb,acr),[4,1,1]);
ds1 = ds_2h1v(a);
ds2 = ds_2h2v(a);
ds3 = ds_4h4v(a);
figure(1)
    subplot(4,4,[1 2 5 6]);
    imagesc(a(:,:,2)); title("Original Chrominance [Cb] layer")
    subplot(4,4,3);  imagesc(a(:,:,1));   title("Original Luminance (Y)");
    subplot(4,4,4);  imagesc(ds1(:,:,1)); title("2h1v Ds Y");
    subplot(4,4,7);  imagesc(ds2(:,:,1)); title("2h2v Ds Y");
    subplot(4,4,8);  imagesc(ds3(:,:,1)); title("4h4v Ds Y");
    subplot(4,4,9);  imagesc(a(:,:,2));   title("Original Cb");
    subplot(4,4,10); imagesc(ds1(:,:,2)); title("2h1v Ds Cb");
    subplot(4,4,11); imagesc(ds2(:,:,2)); title("2h2v Ds Cb");
    subplot(4,4,12); imagesc(ds3(:,:,2)); title("4h4v Ds Cb");
    subplot(4,4,13); imagesc(a(:,:,3));   title("Original Cr");
    subplot(4,4,14); imagesc(ds1(:,:,3)); title("2h1v Ds Cr");
    subplot(4,4,15); imagesc(ds2(:,:,3)); title("2h2v Ds Cr");
    subplot(4,4,16); imagesc(ds3(:,:,3)); title("4h4v Ds Cr");
    sgtitle("Downsampling Chrominance Layers of YCbCr Image")
    \end{lstlisting}
\end{codeblock}
\clearpage


\begin{codeblock}{Assignment1\_OBRIEN -- Main Assignment File}
    This file forms the major component of the assignment - in essence the glue holding the image compression component together, and performing the majority of the video processing.
        
    \begin{lstlisting}[style=Matlab-editor,basicstyle=\mlttfamily, caption={}\label{lst:asst}]   
%% Assignment 1
%   Kane O'BRIEN (A1880046)
%   Init Workspace
%   I'm only interested in generating images during the test stages and for
%   the report; otherwise display flag can/should be set false.  

clear all; close all; clc;
display_flag = true; % Turn figure generation = true;

% Custom functions created and should be pathed -
%     rgb2ycb(improc) - Convert RGB (m,n,3) matrix to YCbCr
%     rgb2ycb2(improc)- Convert RGB (m,n,3) matrix to YCbCr, return three colorised independent channels
%     ycb2rgb(improc) - Convert YCbCr (m,n,3) matrix to RGB
%     rgb2y(imrpoc)   - Reduce the RGB (m,n,3) matrix to luminance (m,n,1) only
%     ds_2h1v(improc) - Downsample Chrominance (m,n,3) layers 2h1v 
%     ds_2h2v(improc) - Downsample Chrominance (m,n,3) layers 2h2v
%     ds_4h4v(improc) - Downsample Chrominance (m,n,3) layers 4h4v
%     ds_dct(improc)  - DCT quantisation


% Quantisation Matrix from Notes 
%{
Note the quantisation table requires image (0,256) and my standard 
luminance values are (0,1). Therefore the ds_dct.m script performs a 
pre-post multiplication of the image. this script only sees image (0,1)
Q= [8  7  7  7  8  9  11 13; ...
    7  7  7  7  8  9  11 14; ...
    7  7  8  8  9  11 13 16; ...
    7  7  8  10 12 14 16 20; ...
    8  8  9  12 15 18 22 26; ...
    9  9  11 14 18 23 29 36; ...
    11 11 13 16 22 29 38 49; ...
    13 14 16 20 26 36 49 65];
%}

img = imread("Ikara.tiff", "tiff");
% This is a large raw image converted to tiff; for sanity, crop the image
% to a workable resolution. Cropping first saves memory and flops.
[x_c,y_c,~] = size(img);
x_c = x_c/2; y_c = y_c/2;

im_crop = img((x_c-240:x_c+239),(y_c-320:y_c+319),:);
xx = [x_c-240, x_c+239, x_c+239, x_c-240, x_c-240]';
yy = [y_c-320, y_c-320, y_c+319, y_c+320, y_c-320]';
im_proc = (double(im_crop)/256); % Bring integer to double precision \in (0,1)

%% Part 1: Image Compression

% Use custom functions to convert RGB => YCbCr and YCbCr => RGB;
ycb = rgb2ycb(im_proc);
rgb = ycb2rgb(ycb);

% Separate layers
lum=ycb(:,:,1); cb=ycb(:,:,2); cr=ycb(:,:,3); % Grayscale YCbCr
% (individual channels
[lumc,cbc,crc] = rgb2ycb2(im_proc); % Colourised YCbCr

% Use custom functions to downsample YCB Chrominance
im_2h1v = ycb2rgb(ds_2h1v(ycb));
im_2h2v = ycb2rgb(ds_2h2v(ycb));
im_4h4v = ycb2rgb(ds_4h4v(ycb));

% Use a custom DCT function
im_ds_dct = ds_dct(ycb);

%% Part 2 - Video Compression
%   My Student ID - a1880046
%   therefore:
%   Fast search algorithm is N-steps-Search
%   Reference frame is 009, prediction frame 010.
%   Appended "bar" vectors for the current processing;
%       "hat" vectors for the best estimate

ref_frame  = imread("video/Car009.tiff", "tiff"); 
pred_frame = imread("video/Car010.tiff", "tiff"); 

im_ref  = rgb2y(double(ref_frame)/256);
im_pred = rgb2y(double(pred_frame)/256);
sad_ops=0;
tic
sad_0 = imsad(im_ref,im_pred); % 0-search - 'Do Nothing' error.
sad_ops = sad_ops+1;
t=toc;

fprintf("Complete 'Do Nothing' Search;\nTime: %2.3f seconds\nNumber of SAD Operations: %d\nSum Of Absolute Differences: %d\n\n",t,sad_ops,sad_0);

%% Exhaustive search
tic
step_ref    = 16;   % Reference Macro Block
step_ser    = 32;   % Search Macro Block
step_ex     = 1;    % Exhaustive Steps
sad_ops     = 0;    % Sum Abs Diff function calls (operations)
x_hat=[]; y_hat=[]; sad_hat=[]; % Best Vector Movement Per Block

[x,y] = size(im_ref);
predicted_frame_exs = ones(x,y);

for i_ = 1:step_ref:x-step_ref % Create Macroblocks
    for j_ = 1:step_ref:y-step_ref 

        % Determine constrained limits for search block; 
        i_l = i_-step_ser;          if i_l <= 1; i_l = 1; end
        i_u = i_+step_ref+step_ser; if i_u >= x; i_u = x; end
        j_l = j_-step_ser;          if j_l <= 1; j_l = 1; end
        j_u = j_+step_ref+step_ser; if j_u >= y; j_u = y; end

        % Crop macroblocks
        macro_ref  = im_ref(i_:i_+step_ref, j_:j_+step_ref);
        macro_pred = im_pred(i_l:i_u-1,j_l:j_u-1); 

        % Perform the search
        x_bar=[0]; y_bar=[0]; sad_bar=[inf]; % Current best guess
        [xx,yy] = size(macro_pred);

        for ii = 1:step_ex:xx-step_ref % Exhaustive search for Ref MB in Search MB
            for jj = 1:step_ex:yy-step_ref

                %Calculate SAD
                sad = imsad(macro_ref, macro_pred(ii:ii+step_ref, jj:jj+step_ref));
                sad_ops = sad_ops +1;  % increment number of SAD operations

                if sad < sad_bar % Better approximation found
                    sad_bar = sad;
                    x_bar   = ii;
                    y_bar   = jj;
                end
            end
        end
        x_hat   = [x_hat; x_bar]; % Save the macroblock-movement vector.
        y_hat   = [y_hat; y_bar];
        sad_hat = [sad_hat;sad_bar]; % Save the macroblock sad (might be useful; ie sum(sad_hat)=total sad.
        predicted_frame_exs(i_:i_+step_ref, j_:j_+step_ref) =  macro_pred(x_bar:x_bar+step_ref, y_bar:y_bar+step_ref);
    end
end
t=toc;
Exhaustive_MacroBlocks = [x_hat y_hat sad_hat];
sad_1_ = sum(sad_hat)
sad_1 = imsad(im_pred,predicted_frame_exs);
fprintf("Complete Exhaustive Search;\nTime: %2.3f seconds\nNumber of SAD Operations: %d\nSum Of Absolute Differences: %d\n\n",t,sad_ops,sad_1_);

%% N-Step-Search Optimised Search Algorithm
tic
step_ref    = 16;   % Reference Macroblock
step_ser    = 32;   % Search Macroblock
step_nss    = 32;   % NSS Init Steps
sad_ops     = 0;    % Sum Abs Diff function calls (operations)
x_hat=[]; y_hat=[]; sad_hat=[]; % Best Vector Movement Per Block

[x,y] = size(im_ref);
predicted_frame_nss = zeros(x,y);

for i_ = 1:step_ref:x-step_ref % Create Macroblocks
    for j_ = 1:step_ref:y-step_ref 

        % Determine constrained limits for search block; 
        i_l = i_ - step_ser;            if i_l <= 1; i_l = 1; end
        i_u = i_ + step_ref + step_ser; if i_u >= x; i_u = x; end
        j_l = j_ - step_ser;            if j_l <= 1; j_l = 1; end
        j_u = j_ + step_ref + step_ser; if j_u >= y; j_u = y; end

        % Crop macroblocks
        macro_ref  = im_ref(i_:i_+step_ref-1, j_:j_+step_ref-1);
        macro_pred = im_pred(i_l:i_u-1,j_l:j_u-1);

        % Perform the search
        [Xm,Ym] = size(macro_pred); % Absolute Search Size

        xx = step_ser; yy = step_ser; % Center Pixel Reference: Errors
        x_bar=[0]; y_bar=[0]; sad_bar=[inf]; % Current best guess
        step_size = step_nss; % Reinitialise the stepsize

        while step_size > 1 % repeat until the step reaches 1
                
            % Jump in the cardinal directions N,E,S,W; limited to available
            % region of search block (funny edge behaviour; non-optimal)

            % North
            xxNmin = xx+step_size; xxNmax = xx+step_ref+step_size-1;
            if xxNmin < 1;      xxNmin = 1;             xxNmax = step_ref;
            elseif xxNmax > Xm; xxNmin = Xm-step_ref+1; xxNmax = Xm; end
            yyNmin = yy; yyNmax = yy+step_ref-1;
            if yyNmin < 1;      yyNmin = 1;             yyNmax = step_ref;
            elseif yyNmax > Ym; yyNmin = Ym-step_ref+1; yyNmax = Ym; end

            % South
            xxSmin = xx-step_size; xxSmax = xx+step_ref-step_size-1;
            if xxSmin < 1;      xxSmin = 1;             xxSmax = step_ref;
            elseif xxSmax > Xm; xxSmin = Xm-step_ref+1; xxSmax = Xm; end 
            yySmin = yy; yySmax = yy+step_ref-1;
            if yySmin < 1;      yySmin = 1;             yySmax = step_ref;
            elseif yySmax > Ym; yySmin = Ym-step_ref+1; yySmax = Ym; end

            % East
            xxEmin = xx; xxEmax = xx+step_ref-1;
            if xxEmin < 1;      xxEmin = 1;             xxEmax = step_ref;
            elseif xxEmax > Xm; xxEmin = Xm-step_ref+1; xxEmax = Xm; end
            yyEmin = yy+step_size; yyEmax = yy+step_size+step_ref-1;
            if yyEmin < 1;      yyEmin = 1;             yyEmax = step_ref;
            elseif yyEmax > Ym; yyEmin = Ym-step_ref+1; yyEmax = Ym; end
            % West
            xxWmin = xx; xxWmax = xx+step_ref-1;
            if xxWmin < 1;      xxWmin = 1;             xxWmax = step_ref;
            elseif xxWmax > Xm; xxWmin = Xm-step_ref+1; xxWmax = Xm; end
            yyWmin = yy-step_size; yyWmax = yy-step_size+step_ref-1;
            if yyWmin < 1;      yyWmin = 1;             yyWmax = step_ref;
            elseif yyWmax > Ym; yyWmin = Ym-step_ref+1; yyWmax = Ym; end

    % Calcuate MacroBlock SAD
    sadN= imsad(macro_ref,macro_pred((xxNmin):(xxNmax),(yyNmin):(yyNmax)));
    sadS= imsad(macro_ref,macro_pred((xxSmin):(xxSmax),(yySmin):(yySmax)));
    sadE= imsad(macro_ref,macro_pred((xxEmin):(xxEmax),(yyEmin):(yyEmax)));
    sadW= imsad(macro_ref,macro_pred((xxWmin):(xxWmax),(yyWmin):(yyWmax)));
    sad_ops = sad_ops+4;

            [sad_bar,Idx] = min([sadN,sadS,sadE,sadW]);
            switch Idx
                case 1 % North
                    xx = xxNmin;
                    yy = yyNmin;
                case 2 % South
                    xx = xxSmin;
                    yy = yySmin;
                case 3 % East
                    xx = xxEmin;
                    yy = yyEmin;
                case 4 % West
                    xx = xxWmin;
                    yy = yyWmin;
            end
                    % 2^n step-division
                    step_size = floor(step_size/2);
        end
        x_hat   = [x_hat; xx];
        y_hat   = [y_hat; yy];
        sad_hat = [sad_hat; sad_bar];
        predicted_frame_nss(i_:i_+step_ref-1, j_:j_+step_ref-1) = macro_pred( (xx):(xx+step_ref-1), (yy):(yy+step_ref-1));
    end
end

t=toc;
NSteps_MacroBlocks = [x_hat y_hat sad_hat];
sad_2  = sum(sad_hat);
sad_2_ = imsad(im_pred,predicted_frame_nss);
fprintf("Complete N-Steps Search;\nTime: %2.3f seconds\nNumber of SAD Operations: %d\nSum Of Absolute Differences: %d\n\n",t,sad_ops,sad_2);

%% Display Images;

if display_flag == true
figure(1)
    subplot(2,2,1)
        imshow(img)
        hold on
        line(yyline,xxline,'Color','red','LineWidth',2)
        title("Imported Image")
        hold off
    subplot(2,2,2)
        imshow(im_crop)
        title("Cropped Image")
    subplot(2,2,[3 4])
        imshowpair(ycb,rgb,'montage')
        title("RGB > YCBCR | YCBCR > RGB")
    sgt=sgtitle("Original Image Processing");
    sgt.FontSize=20;

figure(2)
    subplot(1,3,1)
        imshow(lumc)
        title("Luminance")
    subplot(1,3,2)
        imshow(cbc)
        title("Chrominance Blue")
        impixelinfo
    subplot(1,3,3)
        imshow(crc)
        title("Chrominance Red")
        impixelinfo
    sgt = sgtitle("YCbCr Channels, Colourised through rgb2ycb2");
    sgt.FontSize = 20;

figure(3)
    subplot(2,2,1)
        imshow(im_crop)
        title("Original Image")
    subplot(2,2,2)
        imshow(im_2h1v)
        title("2h1v Downsampled Chrominance")
    subplot(2,2,3)
        imshow(im_2h2v)
        title("2h2v Downsampled Chrominance")
    subplot(2,2,4)
        imshow(im_4h4v)
        title("4h4v Downsampled Chrominance")
    sgt = sgtitle("Images with Chrominance Downsample");
    sgt.FontSize = 20;

figure(4)
    subplot(1,2,1)
       imshow(lumc)
       title("Original Luminance Channel")
    subplot(1,2,2)
        imshow(im_ds_dct(:,:,1))
        title("Luminance Channel DCT Compressed")
    sgt = sgtitle("Images with Luminance-DCT Compression");
    sgt.FontSize = 20;

figure(5)
    subplot(2,2,1)
        imshow(im_ref)
        title("Reference Frame")
    subplot(2,2,2)
        imshow(im_pred)
        title("Predicted Frame")
    subplot(2,2,3)
        imshow(predicted_frame_exs)
        title("Predicted Frame - Exhaustive Search")
    subplot(2,2,4)
        imshow(predicted_frame_nss)
        title("Predicted Frame - n-Step Search")
    sgt = sgtitle("Inter-frame Compression");
    sgt.FontSize = 20;

figure(6)
        subplot(1,2,1)
        imshow(predicted_frame_exs)
        title("Predicted Frame - Exhaustive Search")
    subplot(1,2,2)
        imshow(predicted_frame_nss)
        title("Predicted Frame - n-Step Search")
    sgt = sgtitle("Inter-frame Compression");
    sgt.FontSize = 20;
end  
    \end{lstlisting}
\end{codeblock}