\begin{question}{Question 1:}
  here we ask a question, in the \textbf{``question''} environment.
\end{question}

\begin{answer}{Answer 1:}
Here we answer the question, in the \textbf{``answer''} environment.
\end{answer}

\begin{example}{Example 1:}
Here is an example in a coloured block. This environment is \textbf{``example''.}
\end{example}

\begin{title1}{1}
Here is another example in a different coloured block. This environment is \textbf{title1} as an example without a title.
\end{title1}


\begin{codeblock}{rgb2ycb2.m -- RGB => YCbCr Colourised Transform}
  This function provides colourised conversion of an RGB to YCbCr conversion of a (m,n,3) RGB double matrix.
  Usual chrominance layers are one-dimensional and effectively show grayscale images; this script shows  false coloured chrominance to highlight the differences
      
  \begin{lstlisting}[style=Matlab-editor,basicstyle=\mlttfamily, caption={ttt(improc)}\label{lst:ttt}]
      INSERT CODE HERE

  \end{lstlisting}
\end{codeblock}

\begin{codeblock}{Example Codeblock for MATLAB}
Here is some example code you might like to work with; this code does nothing important :) :) :) :) :) :) :)\\
This is the \textbf{``codeblock"} environment. below is a listing using style=matlab-editor from the \verb|\usepackage{matlab-prettifier}| package

Listing \ref{lst:Matlab} shows an interesting example of code within matlab.
\begin{lstlisting}[style=Matlab-editor,basicstyle=\mlttfamily,caption={MATLAB sample} \label{lst:Matlab}]
%% displaying some quantisation error
signedness=1;   % 1=signed, 0=unsigned;
intbits=5;      % 2^5   = 32
fractbits=1;   % 2^-10 =
wordLen=signedness+intbits+fractbits
q = fixed.Quantizer(signedness,wordLen,fractbits, ...
  'Nearest','Saturate')

for x=0:1:20
  sin(x)+1
end

t=0:.01:100;
x=(10*sin(t)).*(3*cos(2*t));
x_f = fi(x,signedness,wordLen,fractbits);


%{
  This is just a multiline
  comment to see this command works
%}
\end{lstlisting}
\end{codeblock}



\begin{codeblock}{Example Codeblock for Python}
In listing \ref{lst:Python} the python language pack for listing is explored.

\begin{lstlisting}[language=python,basicstyle=\mlttfamily,caption={Python Sample} \label{lst:Python}]
string1 = "Kane"
string2 = "O'Brien"
joined_string = string1 + string2
print(joined_string)
number = 70
# Check the is more than 70 or not
if (number >= 70):
    print("The number is atleast 70")
else:
    print("The number is less than 70")
\end{lstlisting}
\end{codeblock}
\clearpage
